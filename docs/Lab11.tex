%% AMS-LaTeX Created with the Wolfram Language for Students - Personal Use Only : www.wolfram.com

\documentclass{article}
\usepackage{amsmath, amssymb, graphics, setspace}

\newcommand{\mathsym}[1]{{}}
\newcommand{\unicode}[1]{{}}

\newcounter{mathematicapage}
\newcounter{mathematicalastpage}
\begin{document}

1. Compute the sum of the reciprocals of 3, 5, 7, 9, $\ldots $ , 63.\\


\begin{doublespace}
\noindent\(\pmb{\text{listreciprocals}=\text{Table}[1/i,\{i,3,63,2\}]}\)
\end{doublespace}

\begin{doublespace}
\noindent\(\left\{\frac{1}{3},\frac{1}{5},\frac{1}{7},\frac{1}{9},\frac{1}{11},\frac{1}{13},\frac{1}{15},\frac{1}{17},\frac{1}{19},\frac{1}{21},\frac{1}{23},\frac{1}{25},\frac{1}{27},\frac{1}{29},\frac{1}{31},\frac{1}{33},\frac{1}{35},\frac{1}{37},\frac{1}{39},\frac{1}{41},\frac{1}{43},\frac{1}{45},\frac{1}{47},\frac{1}{49},\frac{1}{51},\frac{1}{53},\frac{1}{55},\frac{1}{57},\frac{1}{59},\frac{1}{61},\frac{1}{63}\right\}\)
\end{doublespace}

\begin{doublespace}
\noindent\(\pmb{\text{Total}[\text{listreciprocals}]}\)
\end{doublespace}

\begin{doublespace}
\noindent\(\frac{31674468729962723297623231}{18472920064106597929865025}\)
\end{doublespace}

2. Compute \(\frac{1}{1+\frac{1}{1+\frac{1}{1+\frac{1}{2}}}}\)\\


\begin{doublespace}
\noindent\(\pmb{\text{convergentlist}=\text{Convergents}[\{0,1,1,1,2\}];}\)
\end{doublespace}

\begin{doublespace}
\noindent\(\pmb{\text{Last}[\text{convergentlist}]}\)
\end{doublespace}

\begin{doublespace}
\noindent\(\frac{5}{8}\)
\end{doublespace}

3. Obtain a 50 digit approximation to \(\sqrt{\pi }\)\\


\begin{doublespace}
\noindent\(\pmb{N[\text{Sqrt}[\text{Pi}],50]}\)
\end{doublespace}

\begin{doublespace}
\noindent\(1.7724538509055160272981674833411451827975494561224\)
\end{doublespace}

4. Use the Table function to make a list of the 30 prime numbers starting with 11 and ending with 139. Name the list P.\\


\begin{doublespace}
\noindent\(\pmb{\text{Solve}[\text{Prime}[x]\text{==}11,x]}\)
\end{doublespace}

\begin{doublespace}
\noindent\(\{\{x\to 5\}\}\)
\end{doublespace}

\begin{doublespace}
\noindent\(\pmb{\text{Solve}[\text{Prime}[x]==139,x]}\)
\end{doublespace}

\begin{doublespace}
\noindent\(\{\{x\to 34\}\}\)
\end{doublespace}

\begin{doublespace}
\noindent\(\pmb{p=\text{Table}[\text{Prime}[i],\{i,5,34\}]}\)
\end{doublespace}

\begin{doublespace}
\noindent\(\{11,13,17,19,23,29,31,37,41,43,47,53,59,61,67,71,73,79,83,89,97,101,103,107,109,113,127,131,137,139\}\)
\end{doublespace}

5. Add the numbers in the sequence P found in question 4.\\


\begin{doublespace}
\noindent\(\pmb{\text{Total}[p]}\)
\end{doublespace}

\begin{doublespace}
\noindent\(2110\)
\end{doublespace}

6. Multiply the numbers in the list P found in question 4. Then get a count of the number of digits in the product.\\


\begin{doublespace}
\noindent\(\pmb{\text{lengthp}=\text{Length}[p]}\)
\end{doublespace}

\begin{doublespace}
\noindent\(30\)
\end{doublespace}

\begin{doublespace}
\noindent\(\pmb{\text{productp}=\text{Product}[p[[i]],\{i,1,\text{lengthp}\}]}\)
\end{doublespace}

\begin{doublespace}
\noindent\(47688793574281857464329681579285428844601593586668729\)
\end{doublespace}

\begin{doublespace}
\noindent\(\pmb{\text{IntegerLength}[\text{productp}]}\)
\end{doublespace}

\begin{doublespace}
\noindent\(53\)
\end{doublespace}

7. Sketch the graphs of y = sin [x], y = sin [2x], and y = sin [3x], 0 $\leq $ x $\leq $ 2$\pi $ in steps of $\pi $/2, on one set of axes. Use different
colors for each curve.\\


\begin{doublespace}
\noindent\(\pmb{\text{Plot}[\{\text{Sin}[x],\text{Sin}[2x],\text{Sin}[3x]\},\{x,0,2\text{Pi}\},\text{PlotStyle}\to \{\text{Red},\text{Green},\text{Blue}\},}\\
\pmb{\text{PlotLegends}\to \text{{``}Expressions{''}},\text{Ticks}\to \{\{0,\text{Pi}/2,\text{Pi},3\text{Pi}/2,2\text{Pi}\},\{-1,1\}\}]}\)
\end{doublespace}

\begin{doublespace}
\noindent\(\begin{array}{cc}
  &  \\
\end{array}\)
\end{doublespace}

8. Make a list of the cubes of the integers 2, 5, 6, 9, 12, 44. Add the numbers in the list of cubes and then display the prime factorization of
the sum. What do you notice about the prime factorization? \\


\begin{doublespace}
\noindent\(\pmb{f[\text{x$\_$}]\text{:=}x{}^{\wedge}3}\)
\end{doublespace}

\begin{doublespace}
\noindent\(\pmb{\text{cubeslist}=\text{Map}[f,\{2,5,6,9,12,44\}]}\)
\end{doublespace}

\begin{doublespace}
\noindent\(\{8,125,216,729,1728,85184\}\)
\end{doublespace}

\begin{doublespace}
\noindent\(\pmb{\text{Total}[\text{cubeslist}]}\)
\end{doublespace}

\begin{doublespace}
\noindent\(87990\)
\end{doublespace}

\begin{doublespace}
\noindent\(\pmb{\text{FactorInteger}[\text{Total}[\text{cubeslist}]]}\)
\end{doublespace}

\begin{doublespace}
\noindent\(\{\{2,1\},\{3,1\},\{5,1\},\{7,1\},\{419,1\}\}\)
\end{doublespace}

All prime factors appear only once.

9. Obtain the prime factorization of the product of the integers in the list of cubes described in Exercise 8.\\


\begin{doublespace}
\noindent\(\pmb{\text{FactorInteger}[\text{Product}[\text{cubeslist}[[i]],\{i,1,6\}]]}\)
\end{doublespace}

\begin{doublespace}
\noindent\(\{\{2,18\},\{3,12\},\{5,3\},\{11,3\}\}\)
\end{doublespace}

10. Find two ways to find an approximate value of x for which \(2^x\)= 100 . Display the solution in a graph\\


Method 1

\begin{doublespace}
\noindent\(\pmb{\text{NSolve}\left[2^x-100==0,x,\text{Reals}\right]}\)
\end{doublespace}

\begin{doublespace}
\noindent\(\{\{x\to 6.64386\}\}\)
\end{doublespace}

Method 2

\begin{doublespace}
\noindent\(\pmb{\text{Log}[2,100.]}\)
\end{doublespace}

\begin{doublespace}
\noindent\(6.64386\)
\end{doublespace}

Graph

\begin{doublespace}
\noindent\(\pmb{\text{pointten}=\text{ListPlot}[\{\{6.64386,100\}\},\text{PlotStyle}\to \text{Red},\text{PlotLegends}\to \{\text{{``}(6.64386,100){''}}\}];}\)
\end{doublespace}

\begin{doublespace}
\noindent\(\pmb{\text{plotten}=\text{Plot}[\{2{}^{\wedge}x,y=100\},\{x,0,10\}, \text{AxesLabel}\to \{\text{x},\text{y}\}, \text{PlotLegends}\to \text{{``}Expressions{''}}];}\)
\end{doublespace}

\begin{doublespace}
\noindent\(\pmb{\text{Show}[\text{plotten},\text{pointten}]}\)
\end{doublespace}

\begin{doublespace}
\noindent\(\begin{array}{cc}
  &  \\
\end{array}\)
\end{doublespace}

11. What is the 115th Fibonacci number? What is the 1,115th Fibonacci number?\\


\begin{doublespace}
\noindent\(\pmb{\text{Fibonacci}[115]}\)
\end{doublespace}

\begin{doublespace}
\noindent\(483162952612010163284885\)
\end{doublespace}

\begin{doublespace}
\noindent\(\pmb{\text{Fibonacci}[1115]}\)
\end{doublespace}

\begin{doublespace}
\noindent\(46960625891577894920915085010622289470462518359149677075881383631822660890718642869603700018836567361824279444479341088310462978732670769895389845153583927059046832024176024794070671098298816588315827802770672734166457585412100971385\)
\end{doublespace}

12. What are the greatest common divisor and least common multiple of 5,355 and 40,425?\\


\begin{doublespace}
\noindent\(\pmb{\text{GCD}[5355,40425]}\)
\end{doublespace}

\begin{doublespace}
\noindent\(105\)
\end{doublespace}

\begin{doublespace}
\noindent\(\pmb{\text{LCM}[5355,40425]}\)
\end{doublespace}

\begin{doublespace}
\noindent\(2061675\)
\end{doublespace}

13. Find two ways to compute the sum of the squares of the first 20 consecutive positive integers.\\


Method 1

\begin{doublespace}
\noindent\(\pmb{\text{squareslist}=\text{Table}[i{}^{\wedge}2,\{i,1,20\}]}\)
\end{doublespace}

\begin{doublespace}
\noindent\(\{1,4,9,16,25,36,49,64,81,100,121,144,169,196,225,256,289,324,361,400\}\)
\end{doublespace}

\begin{doublespace}
\noindent\(\pmb{\text{Total}[\text{squareslist}]}\)
\end{doublespace}

\begin{doublespace}
\noindent\(2870\)
\end{doublespace}

Method 2

\begin{doublespace}
\noindent\(\pmb{\text{Sum}[x{}^{\wedge}2,\{x,0,20\}]}\)
\end{doublespace}

\begin{doublespace}
\noindent\(2870\)
\end{doublespace}

14. Find the first three positive solutions to the equation cos(x) = x tan(x). Display your solutions in a graph.\\


\begin{doublespace}
\noindent\(\pmb{\text{FindRoot}[\text{Cos}[x]\text{==}x*\text{Tan}[x],\{x,1\}]}\)
\end{doublespace}

\begin{doublespace}
\noindent\(\{x\to 0.760807\}\)
\end{doublespace}

\begin{doublespace}
\noindent\(\pmb{\text{FindRoot}[\text{Cos}[x]\text{==}x*\text{Tan}[x],\{x,2\}]}\)
\end{doublespace}

\begin{doublespace}
\noindent\(\{x\to 2.81704\}\)
\end{doublespace}

\begin{doublespace}
\noindent\(\pmb{\text{FindRoot}[\text{Cos}[x]\text{==}x*\text{Tan}[x],\{x,6\}]}\)
\end{doublespace}

\begin{doublespace}
\noindent\(\{x\to 6.43558\}\)
\end{doublespace}

\begin{doublespace}
\noindent\(\pmb{\text{Plot}[\{\text{Cos}[x],x*\text{Tan}[x]\},\{x,0,3\text{Pi}\},\text{PlotStyle}\to \{\text{Red},\text{Green}\},\text{AxesLabel}\to
\{\text{x},\text{y}\},}\\
\pmb{\text{PlotLegends}\to \text{{``}Expressions{''}}]}\)
\end{doublespace}

\begin{doublespace}
\noindent\(\begin{array}{cc}
  &  \\
\end{array}\)
\end{doublespace}

\(15. \text{Evaluate}\text{   }\left(\frac{1}{1}+\frac{1}{2}+\frac{1}{3}+\frac{1}{4}\right)+\left(\frac{2}{1}+\frac{2}{2}+\frac{2}{3}+\frac{2}{4}\right)+\left(\frac{3}{1}+\frac{3}{2}+\frac{3}{3}+\frac{3}{4}\right)\)

Method 1

\begin{doublespace}
\noindent\(\pmb{\text{listfourreciprocals}=\{1/1,1/2,1/3,1/4\}}\)
\end{doublespace}

\begin{doublespace}
\noindent\(\left\{1,\frac{1}{2},\frac{1}{3},\frac{1}{4}\right\}\)
\end{doublespace}

\begin{doublespace}
\noindent\(\pmb{6*\text{Total}[\text{listfourreciprocals}]}\)
\end{doublespace}

\begin{doublespace}
\noindent\(\frac{25}{2}\)
\end{doublespace}

Method 2

\begin{doublespace}
\noindent\(\pmb{1/1+1/2+1/3+1/4+2/1+2/2+2/3+2/4+3/1+3/2+3/3+3/4}\)
\end{doublespace}

\begin{doublespace}
\noindent\(\frac{25}{2}\)
\end{doublespace}

\end{document}
